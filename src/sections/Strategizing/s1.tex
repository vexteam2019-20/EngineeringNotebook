% !TeX root = ../../main.tex
\documentclass[../../main.tex]{subfiles}
\begin{document}
\section{\textbf{Finding possibilities}: 2019-9-25/26}
In this meeting, we've decided to start building our strategy, GitHub repositories, and creating an overview of the rules, regulations and specifics of the competition.

\subsection{Starting strategy}
We've decided to split our strategy into two parts: the \textbf{automation} period, and the \textbf{driver control} period.

\subsubsection{Autonomous}
During this 15-sec period, we noticed that drivers aren't allowed to touch the robot or control it. We're only allowed to move the blocks on our half, which means that we have a \textbf{strong advantage}, because we can operate without any opponent interference. We'd like to use this advantage to score points, by putting boxes into our safe zone during that time. We don't think that there is much of a point in putting blocks into the towers, seeing as that is just a \textit{temporary} advantage.

\subsection{Starting Robot Design}
We started to plan out our robot base, using some creative thought.

\subsubsection{Wheels}
We've decided that the optimal layout is one with a rectangular base, with 7 wheels. We considered having a 6-wheel design, with two omnidirectional ones, which are not grippy, and the other 4 being grippy wheels, but we progressed to a design that required using omnidirectional tyres for \textit{all} of our wheels, using 7 wheels.

The reason we want to use omnidirectional tyres for all of the tyres is because of how quickly we can change direction with them; with regular wheels, we need to physically turn the robot, changing t5he orientation to drive in a different direction, but with the omnidirectional tyres, we can preserve the robot's orientation, while still driving in a different direction. This will allow for better autonomous, and driver control.

\subsubsection{Motors}
From the beginning, we decided that we needed 3 motors to drive the wheels. As we finished planning the wheels, we decided that that was still the best option for our 7-wheel design; we need 1 for the omnidirectional wheel at the back, and two for each of the 3-wheel sides. In order to power all three wheels, we are going to make a gearing system from the central motor on either side.

\subsubsection{Base}
The base will have to accomodate for two things: the reverse stacker design that we're using and the wheels on the base. Because we have 7 wheels and a smaller box width than the total width of the wheels 

\subsubsection{Pictures}

\subsubsection{Drawings}

\end{document}